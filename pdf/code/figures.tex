% biafra ahanonu
% latex_boilerplate
% updated: 2013.04.30
%_______FIGURE_______
%Macros to help reduce clutter in text and standardize layout of figures
%______NORMAL_______
\newcommand{\lbpfigure}[6]{
%\lbpfigure{image_location}{width}{reference}{title}{description}{page_location}
\begin{figure}[#6]
	%\begin{left}
	\center
	\scalebox{1}{\includegraphics[width=#2,scale=1.0]{#1}}
	\captionsetup{labelformat=empty}
	\caption{\textbf{{\color{red}Figure \ref{fig:#3}} |  #4}\\ #5}
	\label{fig:#3}
	%\end{left}
\end{figure}
}
\newcommand{\lbpblankfigure}[6]{
%\lbpfigure{image_location}{width}{reference}{title}{description}{page_location}
\vfill
{
\centering
\begin{figure}[#6]
	%\begin{left}
	% \center
	\scalebox{1}{\includegraphics[width=#2,scale=1.0]{#1}}
	\captionsetup{labelformat=empty}
	\caption{#5}
	\label{fig:#3}
	%\end{left}
\end{figure}
}
\vfill
}
%______INTEXT_______
\newcommand{\lbpfloat}[6]{
%\lbpfigure{image_location}{width}{reference}{title}{description}{page_location}
\begin{wrapfigure}{#6}{#2}
	\begin{center}
		\scalebox{1}{\includegraphics[width=#2,scale=1.0]{#1}}
	\end{center}
	\captionsetup{labelformat=empty}
	\caption[#4]{\textbf{{\color{red}Figure \ref{fig:#3}} |  #4}\\ #5}
	\label{fig:#3}
\end{wrapfigure}
}
%______WRAP_______
%This figure can wrap to multiple pages
%\lbpfigure{image_location}{width}{reference}{title}{description}{page_location}
\newcommand{\widefig}[6]{
\begin{wrapfigure}{#6}{0.5\textwidth}
	\vspace{-20pt}
	\begin{center}
		\includegraphics[width=#2,scale=1.0]{#1}
	\end{center}
	\vspace{-20pt}
	\singlespace
	\textbf{Figure \ref{fig:#3} |  #4}\\ #5
	\label{fig:#3}
	\vspace{-10pt}
\end{wrapfigure}
}
%_______Multi_Panel_Figure_______
%Keys loaded when pgfkeys is called
\pgfkeys{
	%Call with \pgfkeys{/family, #1}
	%Define a family directory to avoid name-clashes
	/multiFigure/.is family, /multiFigure,
	%key/.estore in = \newVariable
	label/.estore in = \multiFigLabel,
	capTitle/.estore in = \multiFigCapTitle,
	capDes/.estore in = \multiFigCapDesc,
}
%Create a new multi figure command
\newcommand\multifig[9][]{
	%command looks like \multiFigure[key=val,...]{panelTitle}{imageLocation}..
	%example: 
		%\multiFigure[label=11,capTitle=Hello,capDes=World]{fig}{images/Map.jpg}{fig}{images/Map.jpg}{fig}{images/Map.jpg}{fig}{images/Map.jpg}
	%First arg ([key=val,...]) passed to pgfkeys, get new variables back, circumvent the 9 argument rule for \newcommand
	\pgfkeys{/multiFigure, #1}
	\begin{figure*}[ht!]
	\begin{center}
		\subfloat[#2]{\includegraphics[width=.45\textwidth,scale=1.0]{#3}}  
		\ifthenelse{\isempty{#5}}{}{
		\subfloat[#4]{\includegraphics[width=.45\textwidth,scale=1.0]{#5}}
		}\\
		\ifthenelse{\isempty{#7}}{}{
		\subfloat[#6]{\includegraphics[width=.45\textwidth,scale=1.0]{#7}}
		} 
		\ifthenelse{\isempty{#8}}{}{
		\subfloat[#8]{\includegraphics[width=.45\textwidth,scale=1.0]{#9}}
		}
		\captionsetup{labelformat=empty}
		\caption{\textbf{{\color{red}Figure \ref{fig:\multiFigLabel}} |  \multiFigCapTitle}\\ \multiFigCapDesc}
		\label{fig:\multiFigLabel}
	\end{center}
	\end{figure*}	
}
%_______Watermark_______
\newcommand\watermarkpic[1]{
	\AddToShipoutPicture*{\watermarkpicfxn{#1}}
}
\newcommand\watermarkpicfxn[1]{
	\put(-.01in,0){
		\parbox[b][\paperheight]{\paperwidth}{%
			\vfill
			% \centering
			% ,height=\paperheight
			% width=\paperwidth+.02in
			\includegraphics[width=\paperwidth,keepaspectratio]{#1}%
			\vfill
		}
	}
}