%biafra ahanonu
%LaTeX_Boilerplate
%updated: 2012.12.01

\fchapter{TITLE}{SUBTITLE}{SUBSUBTITLE}{}{}
\noindent
%Example of multi-panel figure
\sfigure[label=11,capTitle=Early native American staples,capDes={(a) Pueblo bonito was an.... (b) Located in Peru. (c) The different type of game hunted in the Americas. (d) The cenote, which were seen as sacred by the Maya.}]{fig}{images/pueblobonito.jpg}{fig}{images/chavin.jpg}{fig}{images/nagame.jpg}{fig}{images/cenote.jpg}
%\textbf{a} Pueblo bonito was an.... \textbf{b} Located in Peru. \textbf{c} the different type of game hunted in the Americas. \textbf{d} the cenote, which were seen as sacred by the Maya.
\ffigure{images/Map.jpg}{6in}{Map}{Map of Civilizations Discussed}{Image of the Americas with locations of the cultures discussed in the text. We start with the Algonquin and then discuss the Cahokia, Hisatsinom, Olmec, Maya, Aztec, Chavin, Inca, and Tehuelche.}{ht}
%Text here
Lorem impsum \citep{Nacuzzi:2007}. Lorem impsum \citep{Cooper:1939}. Lorem impsum \citep{Fernandez:1892}. Lorem impsum \citep{carlin2010linguistics}. Lorem impsum \citep{Febre:2005}. Lorem impsum \citep{Guia:Patagonia}. Lorem impsum \citep{Britannica:Tehuelche:Online}. Lorem impsum \citep{algonquina:lengua:online}. Lorem impsum \citep{algonquina:wigwams:online}. Lorem impsum \citep{Advent:2009}. Lorem impsum \citep{Sultzman:1999}. Lorem impsum \fcite{Map}{}\\