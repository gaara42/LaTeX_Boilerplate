%biafra ahanonu
%LaTeX_Boilerplate
%updated: 2012.12.01

%_______FIGURE_______
%Macros to help reduce clutter in text and standardize layout of figures

%_______CITE_______
\newcommand{\fcite}[2]{(Fig. \ref{fig:#1}#2)}

%______NORMAL_______
\newcommand{\ffigure}[6]{
\begin{figure}[#6]
	%\begin{left}
	\center
	\scalebox{1}{\includegraphics[width=#2]{#1}}
	\captionsetup{labelformat=empty}
	\caption{\textbf{Figure \ref{fig:#3} |  #4}\\ #5}
	\label{fig:#3}
	%\end{left}
\end{figure}
}

%______WRAP_______
%This figure can wrap to multiple pages
\newcommand{\wfigure}[6]{
\begin{wrapfigure}{#6}{0.5\textwidth}
	\vspace{-20pt}
	\begin{center}
		\includegraphics[width=#2]{#1}
	\end{center}
	\vspace{-20pt}
	\singlespace
	\textbf{Figure \ref{fig:#3} |  #4}\\ #5
	\label{fig:#3}
	\vspace{-10pt}
\end{wrapfigure}
}

%_______Multi_Panel_Figure_______
%Keys loaded when pgfkeys is called
\pgfkeys{
%Call with \pgfkeys{/family, #1}
%Define a family directory to avoid name-clashes
 /sfigure/.is family, /sfigure,
%key/.estore in = \newVariable
 label/.estore in = \sfigLabel,
 capTitle/.estore in = \sfigCapTitle,
 capDes/.estore in = \sfigCapDesc,
}
%Create a new figure command
\newcommand\sfigure[9][]{
	%command looks like \sfigure[key=val,...]{panelTitle}{imageLocation}..
	%example: 
		%\sfigure[label=11,capTitle=Hello,capDes=World]{fig}{images/Map.jpg}{fig}{images/Map.jpg}{fig}{images/Map.jpg}{fig}{images/Map.jpg}
	%First arg ([key=val,...]) passed to pgfkeys, get new variables back, circumvent the 9 argument rule for \newcommand
	\pgfkeys{/sfigure, #1}
	\begin{figure*}[ht!]
	\begin{center}
		\subfloat[#2]{\includegraphics[width=3.0in]{#3}}  
		\subfloat[#4]{\includegraphics[width=3.0in]{#5}} \\
		\subfloat[#6]{\includegraphics[width=3.0in]{#7}} 
		\subfloat[#8]{\includegraphics[width=3.0in]{#9}}
		\captionsetup{labelformat=empty}
		\caption{\textbf{Figure \ref{fig:\sfigLabel} |  \sfigCapTitle}\\ \sfigCapDesc}
		\label{fig:\sfigLabel}
	\end{center}
	\end{figure*}	
}

%_______TITLES_______
%Section, chapter and other title formatting
%_______CHAPTER_______
\newcommand{\fchapter}[5]{
%Input five different, each can be called with #(number)
%Example \fchapter{Hello}{}{}{World}{}, gives #1=Hello and #4=World
	\newpage\noindent
	%Add a phantom section to allow correct linking in table of contents
	\phantomsection
	\addcontentsline{toc}{section}{#1}	
	\textsc{#2 \textbar\ #3 \textbar\ #4 \textbar\ #5}\
	\begin{spacing}{0}
		\begin{flushright}
			\textsc{\textbf{\Huge#1}}
			\rule{\textwidth}{.5pt}
		\end{flushright}	
	\end{spacing}
	\vspace{26pt}
	\thispagestyle{empty}
	%Redefine header for each chapter
	\chaptermark{#1}	
	\renewcommand{\leftmark}{#1}
}
%_______SECTION_______
\newcommand{\pchapter}[1]{
%Used for each section, see CHAPTER for implementation details
	\newpage\noindent
	\textsc{}\
	\begin{spacing}{0}
		\begin{flushright}
			\textsc{\textbf{\Huge#1}}
			\rule{\textwidth}{.5pt}
		\end{flushright}	
	\end{spacing}
	\vspace{26pt}
	\thispagestyle{empty}
	%Change to current chapter
	\chaptermark{#1}	
	%Redefine header for each chapter
	\renewcommand{\leftmark}{#1}
	\addcontentsline{toc}{chapter}{#1}
}
%_______TITLE_______
\newcommand{\tchapter}[4]{
%Where the first page title goes
	\textsc{}\
	\begin{spacing}{0}
		\noindent
		\textbf{\scalebox{7}{#1}}\\
		\textbf{\huge#4}
	\end{spacing}
		\vfill
		\begin{flushright}		
		\textbf{\textbf{\huge#3} \huge\textbar\ \huge#2}
		\end{flushright}	
	%Remove styling
	\thispagestyle{empty}
	\vspace{-50pt}
}
%______
%Section MACRO
\newcommand{\fsection}[1]{
%Formatting for all chapters standard through this command
%inputs: 1=chapter, 2=date, 3=location
	\section*{
	\vspace{-2em}
		\begin{flushright}
			#1
		\end{flushright}	
		}
	\vspace{-2.5em}
}
%_______GRAND_SECTION_______
\newcommand{\fgsection}[2]{
%Same 
	\newpage\noindent
	%To help proper linking
	\phantomsection
	\addcontentsline{toc}{chapter}{#1}	
	\textsc{#2 AD}\
	\begin{spacing}{0}
		\begin{flushright}
			\textsc{\textbf{\Huge#1}}
			\rule{\textwidth}{.5pt}
		\end{flushright}	
	\end{spacing}
	\thispagestyle{empty}
	%Change section to current
	\sectionmark{#1}
	%Change header/footer section to current
	\renewcommand{\rightmark}{#1 \textbar\ #2 \textsc{AD}}
}
%_______END_______
\newcommand{\fesection}[2]{
%End page formatting, see others for implementation details
	\newpage\noindent
	\textsc{#2}\
	\begin{spacing}{0}
		\begin{flushright}
			\textsc{\textbf{\Huge#1}}
			\rule{\textwidth}{.5pt}
		\end{flushright}	
	\end{spacing}
	\thispagestyle{empty}
	\sectionmark{#1}
	\renewcommand{\rightmark}{#1 \textbar\ #2 \textsc{AD}}
	\addcontentsline{toc}{chapter}{#1}	
}

%_______SEPARATING_LINE_______
%Make a line for separating in-text sections
\newcommand*\sepline{
  \begin{center}
    \rule[1ex]{.5\textwidth}{.5pt}
  \end{center}}
%_______TABLE_OF_CONTENTS_______
%Alter TOC layout here

\let\oldtableofcontents\tableofcontents%remember the definition
\renewcommand\tableofcontents{
  		\oldtableofcontents\thispagestyle{empty}%use the standard toc
}
\makeatletter 
%Change the TOC command
\renewcommand\tableofcontents{
	%TOC title here
	\begin{spacing}{0}
		\begin{flushright}
			\textsc{\textbf{TOC TITLE}}
			\rule{\textwidth}{.5pt}
		\end{flushright}	
	\end{spacing}
	\pagestyle{empty} 
	%Actually start the TOC
	\@starttoc{toc}} 
\makeatother 