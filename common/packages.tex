%biafra ahanonu
%LaTeX_Boilerplate
%updated: 2012.12.22
%This document contains all the package loaded, add/remove at will
%_____________________
%Declare type, book is just one
%a4 is better looking, remove if in USA
\documentclass[oneside,12pt,a4paper]{book}%a4paper
%_____________________
%FUNCTIONS
%Use key packages for more advanced macros
%\usepackage{keyval}
\usepackage{pgfkeys}
%If then statements
\usepackage{xifthen}
%Allow parsing and looping over lists
\usepackage{amsmath}
\usepackage{xparse}
% Do math
\usepackage[nomessages]{fp}% http://ctan.org/pkg/fp
%_____________________
%CODING
%Include code
% \usepackage{minted}
\usepackage{listings}
\lstset{frame=tb,
  language=R,
  aboveskip=3mm,
  belowskip=3mm,
  showstringspaces=false,
  columns=flexible,
  basicstyle={\small\ttfamily},
  numbers=none,
  numberstyle=\tiny\color{red},
  keywordstyle=\color{blue},
  commentstyle=\color{gray},
  stringstyle=\color{purple},
  breaklines=true,
  breakatwhitespace=true
  tabsize=3
}
%_____________________
%TABLE
\usepackage{color}
\usepackage{colortbl}
\usepackage{booktabs}
\usepackage{ctable}
%Give color to tables
\usepackage[table]{xcolor}
%Make a table that splits itself across multiple pages
\usepackage{longtable}
%_____________________
%GRAPHICS/IMAGE
\usepackage{graphicx,eso-pic}
\usepackage{stfloats}
\usepackage{float}
\usepackage{subfig}
% Left align and justify captions
\usepackage[font=small,justification=justified,singlelinecheck=false]{caption}
\usepackage{wrapfig}
%Insert PDF
\usepackage{pdfpages}
	%\includepdf[pages={1,3,4}]{myfile.pdf}
%_____________________
%TYPOGRAPHIC MODIFICATIONS
%Enable if want no-indent, space between paragraphs
\usepackage[parfill]{parskip}
%For special symbols
\usepackage{gensymb}
%Set line spacing
\usepackage{setspace}
%Arbitrary sizes for fonts
\usepackage{fix-cm}
%Changings a fonts leading
\usepackage{leading}
%Insertion of pre-formatted text (e.g. code)
\usepackage{verbatim}
%Fix spacing table
\usepackage{booktabs}
%Force to be A4
%\usepackage[a4paper]{geometry}
\usepackage{multicol}
%Multicol separator line
\setlength{\columnseprule}{.5pt}
%Space above multicol
\setlength\multicolsep{0pt}
%\setlength{\parskip}{0pt}
%Change lists
\usepackage{enumitem}
%nolistsep
\setlist{noitemsep,topsep=-1em,parsep=0pt,partopsep=-1em}
%\setenumerate{noitemsep,topsep=0pt,parsep=0pt,partopsep=0pt}
%\setdescription{noitemsep,topsep=0pt,parsep=0pt,partopsep=0pt}
%\setitemize{noitemsep,topsep=0pt,parsep=0pt,partopsep=0pt}
%Split arbitrarily long text to multiple lines
\usepackage{seqsplit}
% Flush footnotes
\usepackage[hang,flushmargin]{footmisc} 
% Lipsum included for filler text
\usepackage{lipsum}
% For strike outs
\usepackage{ulem}
%_____________________
%HEADER
%Manipulate margins
\usepackage{styles/simplemargins}
%Underline headers
\usepackage{underlin}
%Modify headers
\usepackage{fancyhdr}
%Remove numbering for initial pages
\pagestyle{fancy}
%_____________________
%MATH
%Typeset mathematical functions
\usepackage{mathtools}
%_____________________
%INTERACTION
%Clickable table of contents
\usepackage{hyperref}
\hypersetup{
  hidelinks=true,
  colorlinks,
  citecolor=red,
  urlcolor=red,
  linkcolor=red,
}
%_____________________
%FONTS
%___Helvetica, use pdfLaTeX
% \usepackage[T1]{fontenc}
% \usepackage{helvet}	
%Set base font for document to sans-serif
%TURN OFF if using Futura or fontspec with XeLaTeX
% \renewcommand{\familydefault}{\sfdefault}

%___Futura, use XeLaTeX
\usepackage{fontspec}
%Must have Futura installed on system
\setmainfont{Futura Std Book}

%___Unicode
%\usepackage{newunicodechar}
%\newunicodechar{é}{\`{e}}

%___Set Font
%\settextfont{Arial}
%\setlatintextfont{Arial}
%_____________________
%TITLE
\usepackage{titlesec}
%Remove spacing above chapter, for bibliography mainly
%\titlespacing*{\chapter}{0pt}{-40pt}{0pt}
%_____________________
%BIBLIOGRAPHY
%cite articles with \citep{}
\usepackage[numbers,super,sort&compress]{natbib}
%\usepackage{biblatex}
%Rename the bibliography to whatever name you want
\renewcommand\bibname{References}
%__________________
%INDEX
%Used to make indices with: \index{term}
\usepackage{makeidx}
\makeindex